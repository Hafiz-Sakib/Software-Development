\documentclass[12pt]{article}
\usepackage{amsmath, amssymb}

\title{AI-Controlled Smart Traffic Management System}
\author{}
\date{}

\begin{document}

\maketitle

\section*{Logical Analysis Using First-Order Logic (FOL)}

\subsection*{Predicates}
\begin{itemize}
    \item $\text{RoadClosed}(R)$: Road $R$ is closed.
    \item $\text{SignalFunctional}(T)$: Traffic signal $T$ is functional.
    \item $\text{IntersectionCongested}(I)$: Intersection $I$ is congested.
    \item $\text{EmergencyPathClear}(E)$: Emergency vehicle $E$ has a clear path.
    \item $\text{AlternativeRouteExists}(R_1, R_2)$: There exists an alternative route $R_2$ for road $R_1$.
    \item $\text{ManualControlRequired}(I)$: Manual traffic control is required at intersection $I$.
\end{itemize}

\subsection*{Rules}
\begin{enumerate}
    \item $\forall R, T, I \ (\text{RoadClosed}(R) \land \text{Controls}(T, R) \implies \neg \text{SignalFunctional}(T))$
    \item $\forall R, R_2 \ (\text{RoadClosed}(R) \land \text{AlternativeRouteExists}(R, R_2) \implies \text{Reroute}(R, R_2))$
    \item $\forall T, I \ (\neg \text{SignalFunctional}(T) \land \text{Controls}(T, I) \implies \text{IntersectionCongested}(I))$
    \item $\forall I, R \ (\text{IntersectionCongested}(I) \land \text{Connected}(R, I) \implies \text{Congested}(R))$
    \item $\forall E, I \ (\text{EmergencyPathClear}(E) \implies \neg \text{IntersectionCongested}(I))$
\end{enumerate}

\subsection*{Incident Conditions}
\begin{itemize}
    \item $\text{RoadClosed}(R1)$
    \item $\text{Controls}(T1, R1)$
    \item $\text{Connected}(R1, I1)$
    \item $\text{AlternativeRouteExists}(R1, R2)$
    \item $\text{EmergencyPathClear}(E1)$
\end{itemize}

\section*{Step-by-Step Explanation of AI’s Decision-Making Process}

\begin{enumerate}
    \item \textbf{Determine Signal Functionality}: From Rule 1, since $\text{RoadClosed}(R1)$ and $\text{Controls}(T1, R1)$, it follows that $\neg \text{SignalFunctional}(T1)$.
    \item \textbf{Determine Intersection Congestion}: From Rule 3, since $\neg \text{SignalFunctional}(T1)$ and $\text{Controls}(T1, I1)$, it follows that $\text{IntersectionCongested}(I1)$.
    \item \textbf{Ensure Emergency Vehicle Path}: From Rule 5, to ensure $\text{EmergencyPathClear}(E1)$, we must prevent $\text{IntersectionCongested}(I1)$. Using $R2$, congestion at $I1$ can be reduced.
    \item \textbf{Determine Need for Manual Control}: From Rule 2, since $\text{RoadClosed}(R1)$ and $\text{AlternativeRouteExists}(R1, R2)$, vehicles can be rerouted through $R2$. Manual control is only required if rerouting fails.
    \item \textbf{Impact of Alternative Route}: The presence of $R2$ ensures efficient rerouting, reducing congestion and guaranteeing $E1$'s timely arrival.
\end{enumerate}

\section*{Answers to Objectives}

\begin{enumerate}
    \item \textbf{Will traffic signal $T1$ remain functional or fail?} $T1$ will fail because $R1$ is closed, and $T1$ controls $R1$.
    \item \textbf{Will Intersection $I1$ become congested?} Yes, $I1$ will become congested due to the failure of $T1$.
    \item \textbf{Can emergency vehicle $E1$ reach the hospital in time?} Yes, $E1$ can reach the hospital in time if traffic is rerouted through $R2$.
    \item \textbf{Will vehicles be effectively rerouted, or is manual traffic control required?} Vehicles will be effectively rerouted through $R2$. Manual control is only required if rerouting fails.
    \item \textbf{How does the presence or absence of an alternative route impact the overall outcome?} The presence of $R2$ ensures efficient rerouting, reduces congestion, and guarantees $E1$'s timely arrival. Without $R2$, manual control would be necessary, leading to delays.
\end{enumerate}

\end{document}