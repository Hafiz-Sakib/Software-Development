% Setting up the literature review section
\section*{Review of Related Studies}
This section reviews various studies that have employed deep learning techniques for plant disease classification. The focus is on the methodologies, architectures, and results achieved in these studies, providing a comprehensive understanding of the advancements in this field.
% Adding a brief introduction to the literature review
% Using enumerate for numbered references
\begin{enumerate}[label=\arabic*.]
  \item \textbf{Yoon et al. (2020) - Unsupervised Image Translation for Plant Disease Recognition}

  This study proposed an unsupervised image translation technique using Generative Adversarial Networks (GANs) combined with Deep CNNs to enhance plant disease recognition. The method demonstrated improved accuracy over traditional augmentation techniques by generating synthetic yet realistic images, which helped in better generalization and robustness.

  \item \textbf{Shi et al. (2019) - Global Pooling Dilated Convolutional Neural Network (GPDCNN) for Cucumber Leaf Disease Detection}

  Published in [Journal Name], this research introduced a GPDCNN architecture specifically designed for detecting cucumber leaf diseases. The model achieved an accuracy of 94.65\% and outperformed conventional CNNs due to its enhanced architectural design, particularly the use of dilated convolutions and global pooling layers.

  \item \textbf{Pratap Singh et al. (2019) - Multilayer Convolutional Neural Network (MCNN) for Mango Anthracnose Detection}

  This work utilized a Multilayer CNN (MCNN) for detecting anthracnose disease in mango leaves. The model achieved a high classification accuracy of 97.13\% without requiring manual feature extraction, showcasing the effectiveness of deep learning in automated plant disease detection.

  \item \textbf{V. Singh (2019) - Particle Swarm Optimization (PSO) for Sunflower Leaf Disease Detection}

  Employing Particle Swarm Optimization (PSO) for image segmentation, this study focused on sunflower leaf disease detection. The method achieved an impressive accuracy of 98\% and required minimal parameter tuning, highlighting PSO's strength in optimizing image-based classification tasks.

  \item \textbf{Sachan et al. (2019) - Deep Convolutional Neural Network (DCNN) for Real-Time Corn Plant Disease Recognition}

  This research presented a DCNN model for real-time corn plant disease recognition using the Plant Village dataset. The model achieved an accuracy of 88.46\% without manual preprocessing, leveraging the self-learning capabilities of deep neural networks.

  \item \textbf{Gupta et al. (2019) - Three-Layer CNN for Tomato Leaf Disease Detection}

  Focusing on tomato leaf disease detection, this study used a three-layer CNN and reported variable accuracies (76--100\%) across different disease categories. The results demonstrated the impact of disease complexity on model performance and highlighted the need for tailored approaches for specific diseases.

  \item \textbf{Arsenovic et al. (2016) - Deep CNN Model for Detecting 13 Plant Diseases}

  This pioneering work introduced a deep CNN model trained on the Caffe framework to detect 13 different plant diseases using real-world agricultural images. The model achieved an average accuracy of 96.3\%, validating the effectiveness of deep models on diverse datasets.

  \item \textbf{Khamparia et al. (2019) - Deep Convolutional Encoder Network for Maize Leaf Disease Detection}

  Developed a deep convolutional encoder network for maize leaf disease detection, achieving an accuracy of 97.5\%. The study emphasized the usefulness of deep autoencoder structures for extracting relevant features from plant images, demonstrating their potential in disease classification tasks.
\end{enumerate}

% Adding concluding paragraph
These studies collectively demonstrate that CNN-based models, when combined with proper preprocessing, augmentation, and architecture tuning, can achieve high accuracy in plant disease classification tasks. Inspired by these works, our study compares six well-known deep learning models---Xception, EfficientNetB0, ResNet50, VGG16, DenseNet121, and InceptionV3---on a cassava leaf disease dataset to determine the most effective model for practical deployment.