\section{Conclusion}

The primary goal of this project was to develop a system capable of detecting diseases in cassava leaves while ensuring a smooth and pleasant user experience for farmers. We evaluated six state-of-the-art CNN architectures—Xception, EfficientNetB0, ResNet50, VGG16, DenseNet121, and InceptionV3—alongside our proposed ReXNet150 model. 

On the validation set, Xception achieved an accuracy of 91.3 % with an F1–score of 91.0 %, while EfficientNetB0 reached 91.1 % accuracy and 90.8 % F1–score. ResNet50, DenseNet121, and InceptionV3 delivered moderate performance with accuracies of 85.0 %, 87.0 %, and 86.4 %, and F1–scores of 84.6 %, 86.8 %, and 86.0 %, respectively. VGG16 lagged behind with 68.0 % accuracy and a 67.5 % F1–score. Our ReXNet150 model outperformed all others, achieving a validation accuracy of 94.7 % and an F1–score of 94.9 %.

These results demonstrate that ReXNet150 provides the best balance between classification accuracy and computational efficiency for cassava leaf disease detection. While the high performance of Xception and EfficientNetB0 confirms their suitability for this task, ReXNet150’s superior metrics make it the preferred choice for real-world deployment.

\section{Future Work}

To further enhance and expand this system, we plan to:
\begin{itemize}
    \item \textbf{Fine-tune ReXNet150 further,} exploring additional hyperparameter optimizations and architectural refinements.
    \item \textbf{Investigate newer and lightweight architectures,} aiming to improve the speed–accuracy trade-off.
    \item \textbf{Apply model compression techniques} such as pruning, quantization, and knowledge distillation, enabling real-time inference on mobile and edge devices.
    \item \textbf{Develop a mobile application} for on-the-spot disease diagnosis, empowering farmers without requiring internet connectivity.
    \item \textbf{Expand the dataset} by collecting images from diverse geographical regions and environmental conditions to bolster model generalization.
    \item \textbf{Extend the framework to other crops,} creating a universal plant disease detection platform.
\end{itemize}

This work lays a solid foundation for intelligent, accessible solutions in agricultural disease management. Future researchers and practitioners can build upon our findings to develop even more robust and versatile systems.
